% !TEX root = main.tex
\section{Data Set}
\label{sect:dataset}
They took 90 projects that was present then in the GHTorrent data set~\cite{Gousios2013} and discarded 13 projects due to having less than 50 discussions. In total they selected 77 projects with 102,122 discussions. Then out of this 77 projects, they randomly selected 5 projects. Then they randomly selected 200 discussions from each projects, totaling 1000 discussions for manual classification. Then the 1000 discussions were manually classified by two individual author of the paper. After the manual classification, 967 discussions were taken in which the classification by both author matched. They 1000 discussion they took were randomly taken from five projects. Also they did not discuss or specify any rules for the manual classification of the 1000 discussions. Due to the randomness of the data and the classification choice being very subjective, it is impossible to replicate the data they used. So we decided to contact them for their corpus of 1000 discussions and they provided us with the data. Also the 77 projects they taken were not specified either. So we used their processed data from the GHTorrent database.

\noindent\textbf{discussions.csv}--- contains the 967 sentences that are manually classified by two of the authors of the paper, 226 (23\%) of these refer to design points and 741 (77\%) of them are other issues labeled as general. A sample of the file is shown in Table~\ref{tbl:discussions.csv}
\begin{table*}
	\centering
	\caption{Data sample in discussions.csv}
	\label{tbl:discussions.csv}
	\begin{tabular}{lccc} 
		\toprule
		\textbf{Discussion Text} & \textbf{Label}  \\
		\midrule
		Did you send a bug report comment upstream or something because otherwise we are gonna have to fix this again with the next version of assimp & general \\
		SSL contexts shouldnt be reused across connections see So its probably more appropriate to pass in factories directly & design \\
		\bottomrule
	\end{tabular}
\end{table*}

\noindent\textbf{rq\_1.csv}--- is a data file with 586839 lines of data. Each line represents three columns separated by space. The first column points to the repository and project name separated by `\'. The second column provides information about the whether this discussion comes from an issue, commit or pull request with the associated commit, issue or pull request number. The third column represents the label that the classifier assigned. Table \ref{tbl:rq_1.data} shows a sample of the data that this file contains.
\begin{table*}
	\centering
	\caption{Data sample in rq\_1.data}
	\label{tbl:rq_1.data}
	\begin{tabular}{lccc} 
		\toprule
		\textbf{Repository/Project} & \textbf{Event Category with Number}  & \textbf{Label}  \\
		\midrule
		akka/akka & akka/akka-commit\_comments-d9e0088f3cc537ea342f6fc4e99ee5319dfc94ae & general \\
		akka/akka & akka/akka-commit\_comments-3ce3f270dfce5da7aa5b6270b0559e2c3c0fff6f & design \\
		ariya/phantomjs & ariya/phantomjs-issue\_comments-10045 & design \\
		antirez/redis & antirez/redis-issue\_comments-160 & general \\
		\bottomrule
	\end{tabular}
\end{table*}

\noindent\textbf{rq\_2\_a.data}--- also contains 586839 lines of data divided into four columns by single space that represent username of the contributor, commit/issue/pull request with respective number, label of the discussion and repository name/project name respectively as shown by the sample in Table \ref{tbl:rq_2_a.data}.
\begin{table*}
	\centering
	\caption{Data sample in rq\_2\_a.data}
	\label{tbl:rq_2_a.data}
	\begin{tabular}{lccc} 
		\toprule
		\textbf{Username} & \textbf{Event Category with Number}  & \textbf{Label} & \textbf{Repository/Project}  \\
		\midrule
		viktorklang & akka/akka-commit\_comments-d9e0088f3cc537ea342f6fc4e99ee5319dfc94ae & general & akka/akka \\
		jboner & akka/akka-commit\_comments-a3026b3316dc5b34c3d37ce6fc56cc44bac1d561 & design & akka/akka \\
		antirez & antirez/redis-issue\_comments-646 & general & antirez/redis \\
		patriknw & akka/akka-pull\_request\_comments-149 & general & akka/akka \\
		\bottomrule
	\end{tabular}
\end{table*}

\noindent\textbf{rq\_2\_b.data}--- is a narrowed copy of `rq\_2\_a.data' that contains the username of the contributor, repository/project name and commit, issue or pull\_request with number that points to only the design label.

\noindent\textbf{rq\_2\_a\_c.data}--- hold 23293 lines of data that are divided into three columns. The first column represents the repository name/project name, the second column shows the user name of the contributor and the last column point to the number of design discussions the user in the previous column addressed in the project defined by the first column. Table~\ref{tbl:rq_2_a_c.csv} shows the sample of data in the file.  
\begin{table*}
	\centering
	\caption{Data sample in rq\_2\_a\_c.data}
	\label{tbl:rq_2_a_c.data}
	\begin{tabular}{lccc} 
		\toprule
		\textbf{Repository/Project} & \textbf{Username} & \textbf{Involvement in design discussions}   \\
		\midrule
		zurb/foundation & stewarty & 1 \\
		rails/rails & hassox & 1 \\
		rails/rails & stjhimy & 1 \\
		twitter/finagle & kolbasov & 1 \\
		cakephp/cakephp & atkrad & 1 \\
		zendframework/zf2 & ravids & 1 \\
		symfony/symfony & jdhoek & 6 \\
		joyent/node & paulfryzel & 1 \\
		mxcl/homebrew & hackdefendr & 9 \\
		\bottomrule
	\end{tabular}
\end{table*}

   