% !TEX root = main.tex
\section{Replication}
We strictly followed the steps demonstrated in the paper. While replicating, we took the following steps:

\subsection{Stopwords Removal}
Stopwords are the words that has the similar likelihood of occurring in documents regardless of the relevance of the query~\cite{Wilbur1992}. It is very important to remove stopwords to improve the performance of the classifier as well as reducing the size of the dataset. We have used the The \emph{Natural Language Toolkit's (NLTK)}\footnote{NLTK, https://www.nltk.org/}~\cite{Loper2002} english stopwords set to primarily remove some general stopwords. Then we have looked for some document specific stopwords ex. `lgtm' which is short form of `looks good to me' and removed them. The before and after status of a sentence for stopwords removal can be seen in Table~\ref{tbl:stopwords_removal}
  \begin{table}
  	\caption{Sample of sentence status before and after stopwords removal. The sentence on the top is the actual sentence. The bottom one is after removing the stopwords  }
  	\begin{tabular}{ p{3.25in}}
	 	\toprule
	  		Did you send a bug report comment upstream or something because otherwise we are gonna have to fix this again with the next version of assimp \\
  		\midrule
	  		send bug report comment upstream something otherwise gonna fix next version assimp \\
  		\bottomrule

  	\end{tabular}
  	\label{tbl:stopwords_removal}
  \end{table}   
\label{sect:replication}  
